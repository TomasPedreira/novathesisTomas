%!TEX root = ../template.tex
%%%%%%%%%%%%%%%%%%%%%%%%%%%%%%%%%%%%%%%%%%%%%%%%%%%%%%%%%%%%%%%%%%%%
%% abstract-en.tex
%% NOVA thesis document file
%%
%% Abstract in English([^%]*)
%%%%%%%%%%%%%%%%%%%%%%%%%%%%%%%%%%%%%%%%%%%%%%%%%%%%%%%%%%%%%%%%%%%%

\typeout{NT FILE abstract-en.tex}%

\paragraph{}Over the years, the world of technology has been expanding in every possible field, 
and agriculture is no exception. The use of technology in Smart Agriculture has been increasing, with 
Internet of Things networks, Artificial Intelligence systems for management and forecasting, and robotics to 
automate processes, do the repetitive jobs humans would rather not do, like applying pesticide, 
harvesting products, and evaluating the state of the fields, and also some that humans can't do, 
surveilling large fields. These developments not only come to facilitate tasks but also to battle ongoing problems 
with the agricultural environment, like the lack of young people willing to do hard work on the fields 
and the increase of the world population, that will increase the need for food production. 
This work is focused on the development of 
a tractor-trailer robot, capable of autonomously navigating through a field while spraying pesticides 
directly on the products. The focus is on the navigation side of the robot, researching the use of 
robotics in Smart Agriculture, researching robot path planning methods like Voronoi Graphs, Visibility graphs, Sampling-based approaches like 
the Rapidly exploring Random Trees and the Probabilistic Roadmap, 
Bio-heuristic algorithms like the Ant Colony Optimization, Particle Swam Optimization and the Genetic Algorithm, and also mentioning the learning-based methods. 
This work will also review a few control methods included in the library nav2 of ROS2 like the Dynamic Windows Approach,
 Pure Pursuit, and the Model Predictive Controller. This task is already a challenging one, however, with the addition of the trailer 
 to the system, the task becomes even more challenging, as the trailer dynamics will need to be accounted 
for. In the end the chosen methods to be tested will be the Voronoi Graph with the A* algorithm for path 
planning, and the Pure Pursuit controller for the trajectory tracking.
The tractor-trailer system was chosen, despite its complexity and limited documentation, 
due to its modularity, allowing for independent operation and reusability.

\keywords{
  Smart Agriculture \and
  tractor-trailer \and
  IoT \and
  Artificial Intelligence \and
  Robotics \and
  Path planning \and
  Robot control \and
  ROS2 \and
  Voronoi Graph \and
  A* \and
  Pure Pursuit 
}
